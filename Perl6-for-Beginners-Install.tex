\begin{frame}[fragile]{Getting \& Installation (Binaries)}
\begin{block}<1->{macOS \& Windows}
\begin{itemize}
\item<1-> Go to \url{https://rakudo.org/files}
\item<2-> Choose your Operating System or Windows
\item<3-> Follow the installation instructions for your Operating System or Windows
\end{itemize}
\end{block}
\end{frame}

%\begin{frame}[fragile]{Packages Repository I/II}
%\begin{block}<1->{*BSD, Linux, \ldots - use package repository}
%\begin{lstlisting}[language=sh,inputencoding=latin9]
%nbsd   $ export PKG_PATH="http://cdn.NetBSD.org/pub/\
%         pkgsrc/packages/OPSYS/ARCH/VERSIONS/All/"
%fbsd   $ # pkg knows it's repository
%obsd   $ export PKG_PATH=scp://user@company-build-\
%         server/usr/ports/packages/\%a/all:\
%	 https://trusted-public-server/\%m:\
%	 installpath
%debian $ sudo apt-get -y update
%centos $ # yum updates repositories automatically
%suse   $ sudo zypper refresh
%\end{lstlisting}
%\end{block}
%\end{frame}
%
%\begin{frame}[fragile]{Packages Repository II/II}
%\begin{block}<1->{*BSD, Linux, \ldots - install from package repository}
%\begin{lstlisting}[language=sh,inputencoding=latin9]
%nbsd   $ sudo pkg_add rakudo-star
%fbsd   $ sudo pkg install rakudo-star
%obsd   $ sudo pkg_add rakudo-star
%debian $ sudo apt-get install rakudo-star
%centos $ sudo yum install rakudo star
%suse   $ sudo zypper install rakudo-star
%\end{lstlisting}
%\end{block}
%\end{frame}
%
%\begin{frame}[fragile]{Install from Source}
%\begin{block}<1->{Install Rakudo Star from Source}
%There is an excellent guide at \url{https://rakudo.org/files/star/source}
%for most widely used operating system distributions. \uncover<2->{Since
%there is the source download included, an offline copy doesn't make much
%sense.}
%
%\uncover<3->{\textrightarrow DVD distributors might ask for an offline copy
%for the targeted audience.}
%\end{block}
%\end{frame}
%
%\begin{frame}[fragile]{Manage multiple Rakudo Installations}
%\begin{block}<1->{Brewing continues \ldots}
%\begin{itemize}
%\item<1-> Visit \url{https://github.com/tadzik/rakudobrew}
%\item<2-> Install \texttt{git} unless already available
%\item<3-> run \lstinline[language=sh,inputencoding=latin9]!git clone https://github.com/tadzik/rakudobrew ~/.rakudobrew!
%(Windows users might prefer \lstinline[language=sh,inputencoding=latin9]!%USERPROFILE%\rakudobrew!
%instead of \lstinline[language=sh,inputencoding=latin9]!~/.rakudobrew! in this \uncover<4->{and next command})
%\item<4-> run ~/.rakudobrew/bin/init! as described in \texttt{README.md}, e.g.
%\lstinline[language=sh,inputencoding=latin9]!echo 'eval "$(~/.rakudobrew/bin/rakudobrew init Bash)"' \!
%\lstinline[language=sh,inputencoding=latin9]!  >> ~/.profile!
%\end{itemize}
%\end{block}
%\end{frame}

