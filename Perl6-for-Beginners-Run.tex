\subsection{Running Perl 6}

\begin{frame}[fragile]{Starting Perl 6 Program}
\begin{block}<1->{How to run Perl 6 Programs}
\begin{itemize}
\item<1-> Run explicitely with compiler:
\begin{lstlisting}[language=sh,inputencoding=latin9]
sno@ernie$ perl6 prog.p6
\end{lstlisting}
\item<2-> Run self-contained
\begin{lstlisting}[language=sh,inputencoding=latin9]
sno@ernie$ head -n1 ./prog.p6
#!/opt/pkg/bin/perl6
sno@ernie$ chmod +x ./prog.p6
sno@ernie$ ./prog.p6
\end{lstlisting}
\end{itemize}
\end{block}
\end{frame}

\begin{frame}[fragile]{Starting Program With Parameters}
\begin{block}<1->{How to run Perl 6 Programs With Parameters}
\begin{itemize}
\item<1-> Run explicitely with compiler:
\begin{lstlisting}[language=sh,inputencoding=latin9]
sno@ernie$ perl6 prog.p6 --arg1 --arg2=val pos_arg
\end{lstlisting}
\item<2-> Run self-contained
\begin{lstlisting}[language=sh,inputencoding=latin9]
sno@ernie$ head -n1 ./prog.p6
#!/opt/pkg/bin/perl6
sno@ernie$ chmod +x ./prog.p6
sno@ernie$ ./prog.p6 --arg1 --arg2=val pos_arg
\end{lstlisting}
\end{itemize}
\end{block}
\end{frame}

\begin{frame}[fragile]{Starting Perl6 with Parameters}
\begin{block}<1->{Starting Perl6 with Parameters}
\begin{itemize}
\item<1-> Perl 6 specifies in \url{https://github.com/perl6/specs/blob/master/S19-commandline.pod}
which parameters are available and what they do.
\item<2-> A more detailed \burlalt{https://github.com/perl6/specs/blob/master/S19-commandline.pod#Option-Reference}{Option Reference} is available later on that page - so scroll down and read until the end!
\item<3-> However, Rakudo implements some options (\texttt{-M}) and refuses some options (\texttt{-F}) - so always refer to \lstinline[language=sh,inputencoding=latin9]!perl6 --help!
\item<4-> Change perspective: Some things are better solved differently than adding options \lstinline[language=sh,inputencoding=latin9]!perl6-debug ./prog.p6! replaces \lstinline[language=sh,inputencoding=latin9]!perl -d ./prog.pl!
\end{itemize}
\end{block}
\end{frame}

